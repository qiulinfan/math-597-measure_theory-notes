\documentclass[11pt]{article}
\usepackage{amsmath,amssymb,alltt,fancyheadings, amsfonts}
\usepackage{times, mystyle}
\usepackage{tikz}
\usepackage{enumitem}
\usepackage{hyperref}

\usetikzlibrary{fixedpointarithmetic}

\addtolength{\headheight}{3pt}

\addtolength{\textwidth}{100pt}
\addtolength{\evensidemargin}{-50pt}
\addtolength{\oddsidemargin}{-50pt}

\addtolength{\textheight}{120pt}
\addtolength{\topmargin}{-50pt}

\newcommand{\cA}{\mathcal{A}}
\newcommand{\cB}{\mathcal{B}}
\newcommand{\cP}{\mathcal{P}}
\newcommand{\ba}{\mathbf{a}}
\newcommand{\bi}{\mathbf{i}}
\newcommand{\bj}{\mathbf{j}}
\newcommand{\bk}{\mathbf{k}}
\newcommand{\br}{\mathbf{r}}
\newcommand{\bv}{\mathbf{v}}
\newcommand{\bF}{\mathbf{F}}
\renewcommand{\bS}{\mathbf{S}}
% \newcommand{\GL}{\protect\operatorname{GL}}
\newcommand{\proj}{\protect\operatorname{proj}}
\newcommand{\orth}{\protect\operatorname{orth}}
\newcommand{\curl}{\protect\operatorname{curl}}
\newcommand{\SP}{\protect\operatorname{Sp}}
%\newcommand{\R}{\protect\operatorname{R}}

\pagestyle{empty}
%\setlength{\headrulewidth}{0pt}
%\setlength{\footrulewidth}{0pt}

\begin{document}


\thispagestyle{empty}


%%%%%%%%%%%%%%%%%%%%%%%%%%%%%%%%%%%%%%%%%%%%%%%%%%%%%%%%%%%%


\begin{center}
\textbf{%
Math 597, Winter 2025\\
Homework Set 1\\
}
\end{center}
\begin{center}
\textit{Only some of the questions on this and other homework sets will be graded}.
\end{center}

\begin{enumerate}
  %1.1
\item (Borel vs open.)
  Let $X$ be a metric space such that every subset of $X$ is Borel measurable. Does it follow that every subset of $X$ is open? Give a proof or a counterexample.

\item (Restriction of a $\sigma$-algebra to a subset.)
  Let $X$ be a set, and $Y\subset X$ a subset.
  \begin{itemize}
  \item[(a)]
    Given a $\sigma$-algebra $\cA$ on $X$, prove that 
    \begin{equation*}
      \cA|_{Y}:=\{ E \cap Y\mid E\in \cA\} 
    \end{equation*}
    is a $\sigma$-algebra on $Y$. 
  \item[(b)]
    Given a $\sigma$-algebra $\cB$ on $Y$, prove that there exists a $\sigma$-algebra $\cA$ on $X$ such that $\cA|_Y=\cB$.
  \item[(c)]
    Is the $\sigma$-algebra $\cB$ in~(b) unique? Give a proof or a counterexample.
   \end{itemize}

\item (Invariance properties of the Borel $\sigma$-algebra on $\R^n$).
  \begin{itemize}
    \item[(a)]
      Prove that $\cB(\R^n)$ is translation invariant, i.e. if $A\subset\R^n$ is a Borel measurable set, then
      \[t+A:=\{t+x\mid x\in A\}\]
      is a Borel measurable set for every $t\in \R^n$. 
      \textit{Hint}: For any fixed $t$, show that $\cA=\{B\subset \R^n: t+B\in \cB(\R^n)\}$ is a $\sigma$-algebra.
    \item[(b)]
      Prove that $\cB(\R^n)$ is scaling invariant, i.e. if $A\subset\R^n$ is a Borel measurable set, then \[\lambda A=\{\lambda x\mid x\in A\}\] is a Borel measurable set for every $\lambda\in\R$. 
    \end{itemize}
    
  \item (Hex and such.)
  % \begin{itemize}
  % \item[(a)]
  %   Let $E_1, E_2, \cdots$ be a sequence of subsets of $X$. Show that 
  %   \begin{equation*}
  %     \bigcup_{n=1}^\infty \bigcap_{k=n}^\infty E_k = \{ x\in X: \text{$x\in E_i$ for all but finitely many $i$}\} 
  %   \end{equation*}
  % \item[(b)]
    Let $A\subset [0,1]$ be the set of real numbers in $[0,1]$ having a hexadecimal  expansion with the  digit 5 appearing infinitely many times, and the `digit' E appearing at most finitely many times. Prove that $A$ is a Borel set. \textit{Hint}: see p,2 of Folland's book.
%  \end{itemize}

\item (Admissible annuli.)
  Define an \emph{admissible annulus} in $\R^2$ to be a set of the form
  \begin{equation*}
    \{(x,y)\in\R^2\mid r^2<(x-a)^2+(y-b)^2<R^2\}
  \end{equation*}
  where $a,b\in\Q$, $r,R\in\Q_{>0}$, and $r<R$.
  \begin{itemize}
  \item[(a)]
    Prove that there are only countably many admissible annuli.
  \item[(b)]
    Prove that every open subset of $\R^2$ is a countable union of (not necessarily disjoint) admissible annuli.
  \item[(c)]
    Prove that the Borel $\sigma$-algebra on $\R^2$ is generated by the collection of admissible annuli.
  \end{itemize}
\end{enumerate}

\newpage
\begin{center}
  \textit{Nur f\"ur Verr\"uckte}
\end{center}
(It's \textbf{really} not necessary to attempt these problems. Do not hand them in!)
\begin{enumerate}
\item
  Let $X$ be a set, and define two operations on $\cP(X)$:
  \begin{itemize}
  \item
    the ``product'' of two subsets $E,F\subset X$ is the intersection $E\cap F$;
  \item
    the ``sum'' of two sets $E,F\subset X$ is the symmetric difference $E\Delta F$.
  \end{itemize}
  \begin{itemize}
  \item[(a)]
    Prove that these operations endow $\cP(X)$ with the structure of a commutative ring. What are the additive and multiplicative units? Prove that this ring is idempotent.
  \item[(b)]
    Let us say that a nonempty subset $\cA\subset\cP(X)$ is a \emph{ring} if it is closed under differences and finite unions. In other words, if $E,F\in\cA$, then $E\setminus F\in\cA$ and $E\cup F\in\cA$. Prove that a subset $\cA\subset\cP(X)$ is an algebra iff it is a ring containing $X$.
  \item[(c)]
    Prove that a nonempty subset $\cA\subset\cP(X)$ is a ring iff it is a subring of $\cP(X)$. Also prove that it is an algebra iff it is a subring containing the multiplicative identity.
 \end{itemize}
\item
  Let $(X,\cA)$ and $(Y,\cB)$ be measurable spaces. Say that a map $f:X\to Y$ is \emph{measurable} (with respect to the $\sigma$-algebras $\cA$ and $\cB$) if $f^{-1}(E)\in\cA$ for every $E\in\cB$.
  \begin{itemize}
  \item[(a)]
    Prove that measurable spaces with measurable maps as morphisms form a category.
  \item[(b)]
    Try convincing an analyst that~(a) is useful.
  \end{itemize}
\end{enumerate}
\end{document}  

